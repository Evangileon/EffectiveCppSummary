\documentclass[10pt,landscape]{article}
\usepackage{multicol}
\usepackage{calc}
\usepackage{ifthen}
\usepackage[letterpaper,landscape]{geometry}
\usepackage{hyperref}
\usepackage{titlesec}% http://ctan.org/pkg/titlesec
\usepackage{listings}
\usepackage{color}

% To make this come out properly in landscape mode, do one of the following
% 1.
%  pdflatex latexsheet.tex
%
% 2.
%  latex latexsheet.tex
%  dvips -P pdf  -t landscape latexsheet.dvi
%  ps2pdf latexsheet.ps


\ifthenelse{\lengthtest { \paperwidth = 11in}}
	{ \geometry{top=.5in,left=.5in,right=.5in,bottom=.5in} }
	{\ifthenelse{ \lengthtest{ \paperwidth = 297mm}}
		{\geometry{top=1cm,left=1cm,right=1cm,bottom=1cm} }
		{\geometry{top=1cm,left=1cm,right=1cm,bottom=1cm} }
	}
	
%\geometry{top=1cm,left=1cm,right=1cm,bottom=1cm}

% Turn off header and footer
\pagestyle{empty}
 

% Redefine section commands to use less space
\makeatletter
\renewcommand{\section}{\@startsection{section}{1}{0mm}%
                                {-1ex plus -.5ex minus -.2ex}%
                                {0.5ex plus .2ex}%x
                                {\normalfont\large\bfseries\underline}}
\renewcommand{\subsection}{\@startsection{subsection}{2}{0mm}%
                                {-1explus -.5ex minus -.2ex}%
                                {0.5ex plus .2ex}%
                                {\normalfont\normalsize\bfseries}}
\renewcommand{\subsubsection}{\@startsection{subsubsection}{3}{0mm}%
                                {-1ex plus -.5ex minus -.2ex}%
                                {1ex plus .2ex}%
                                {\normalfont\small\bfseries}}
\makeatother

% Numbering subsections continuously
\let\oldsection\section
\renewcommand{\section}{\oldsection*}
\renewcommand{\thesubsection}{\arabic{subsection}}

% Define BibTeX command
\def\BibTeX{{\rm B\kern-.05em{\sc i\kern-.025em b}\kern-.08em
    T\kern-.1667em\lower.7ex\hbox{E}\kern-.125emX}}

% Don't print section numbers
\setcounter{secnumdepth}{2}


\setlength{\parindent}{0pt}
\setlength{\parskip}{0pt plus 0.5ex}

% Code block configs
\lstset{
  language=C++,
  basicstyle=\linespread{.94}\ttfamily\scriptsize,
  %aboveskip={-0ex},
  %belowskip={-.2ex},
  mathescape=true,
  breaklines=true,
  %frame=single,
  backgroundcolor=\color{white},
}

% Config enumerate
\let\oldenumerate\enumerate
\renewcommand{\enumerate}{
  \oldenumerate
  \setlength{\itemsep}{0pt}
  \setlength{\parskip}{0pt}
  \setlength{\parsep}{0pt}
}


% -----------------------------------------------------------------------

\begin{document}

\raggedright
\footnotesize

\begin{multicols}{3}

% multicol parameters
% These lengths are set only within the two main columns
%\setlength{\columnseprule}{0.25pt}
\setlength{\premulticols}{1pt}
\setlength{\postmulticols}{1pt}
\setlength{\multicolsep}{1pt}
\setlength{\columnsep}{2pt}

\begin{center}
     \Large{\textbf{Effective C++ Cheat Sheet}} \\
\end{center}

\section{Shifting from C to C++}

\subsection{Prefer \texttt{const} and \texttt{inline} to \texttt{\#define}}

\subsection{Prefer \texttt{<iostream>} to \texttt{<stdio.h>}}

\subsection{Prefer \texttt{new} and \texttt{delete} to \texttt{malloc} and \texttt{free}}

\subsection{Prefer C++-style comment}

\section{Memory Management}

\subsection{Use same form of \texttt{new} and \texttt{delete}}
\begin{tabular}{@{}ll@{}}
\texttt{new class} & \texttt{delete *class} \\
\texttt{new class[]} & \texttt{delete [] class} \\
\end{tabular}

\subsection{Use delete on pointer members in dtors}
\label{sec:delete_pointer_in_dtor}
Avoid memory leak. \texttt{delete} a null pointer is always \emph{safe}

\subsection{Be prepared for \textsf{out-of-memory} conditions}
Always check whether the allocation succeed after \texttt{new}, using \texttt{try .. catch} block or \emph{assert NOT} null pointer. \\
This handler invoked while fail to allocate req memory, \texttt{<new>} \\
\begin{lstlisting}
typedef void (*new_handler)();
new_handler set_new_handler(new_handler p) throw();
\end{lstlisting}

\begin{lstlisting}
class X { // specify new_hanlder for a class
public:
  static new_handler set_new_handler(new_handler p);
  static void * operator new(size_t size);
private:
  static new_handler currentHandler;
} // This is also useful for Object Pool
\end{lstlisting}

\subsection{Follow the convention of \texttt{new} and \texttt{delete}}
\label{sec:user_new_delete}
This normal form of this two operators is following \\
\begin{lstlisting}
class Base { // user-define just for this class
  static void * operator new(size_t size);
  static void operator delete(void *p, size_t size);
} // compiler will fill size parameter automatically
\end{lstlisting}

\subsection{You have to write normal form of \texttt{new}}
That means if you want to write a new handler, which has extra parameters other than "normal" form mention in last item, you have to write normal \texttt{new} first.

\subsection{Write \texttt{delete} if you write \texttt{new}}
Once you define your own \texttt{new}, you probably change the behavior of default allocation. For example, if you apply memory pool in allocation function, you should use the pool as proxy to manage the memory allocation and collection. Add this clause to origin defintion of class in Item ~\ref{sec:user_new_delete}: \\
\begin{lstlisting}
static Pool memPool;
\end{lstlisting}
Where memPool is \textsf{non-local static object}. See Item ~\ref{sec:non_local_static_object}

\section{Ctors, Dtors, and =}

\subsection{Declare a copy ctor and = for class with dynamically allocated memory}
Default assignment operator \texttt{=} only flat copy the value of every member variables from right object to left object.
If the object contains pointer variables, issues come out. Because you have to write delete pointer statements in dtors due to Item ~\ref{sec:delete_pointer_in_dtor}. Once one of the two object been deleted, the content that the pointer indicates will be freed. Hence the member pointer of another object will pointer to a memory area that already freed. That probably raise a segment fault exception. \\
In ctor and \texttt{=}, you have three strategy: \\
\begin{tabular}{@{}ll@{}}
\verb!Strategy! & Decription and Example\\
\verb!Alloc new memory & Deep Copy! & STL containers: vector\\
\verb!Reference counter! & STL C++11 shared\_ptr\\
\end{tabular}

\subsection{Prefer initialization to assignment in ctors}
Because construction of objects proceeds in two steps:\\
\begin{enumerate}
  \item Initialization of data members. See Item ~\ref{sec:initialization_list_order}
  \item Execution of the body of ctors
\end{enumerate}
If you write assignment statement in ctor body, (I guess you must want to initialze the object this way), after read two steps above, you will realize that the initalzation operations have been executed twice!\\
That is the reason you should use init list instead of assigns.\\
However, if you have a lot of primitive type members in your class, using assignments is \emph{fine}.\\
Additionally, there is \emph{NO} operational difference between initialization and assignment for non-const, non-reference objects of build-in(primitive) types.

\subsection{List members in an initialization list in the order in which they are declared}
\label{sec:initialization_list_order}
Otherwise the compiler has to keep track of the order in which the members were initialized for \emph{each object}. By the way, The destructor for members are always called in the inverse order of their constructors.

\subsection{Make sure base class have virtual dtor}
When you try to \texttt{delete} a derived class object through a base class pointer and the base class has a \emph{non-virtual} destructor, the results are \emph{undefined}. (Perhaps what happens at runtime is that derived class's destructor is \emph{never called}.

\subsection{Have \texttt{=} return a reference to \texttt{*this}}
The signiture of operator = is as follows, see ~\ref{sec:what_cpp_silently_writes_and_calls} \\
\begin{lstlisting}
C& C::operator= (const C&);  // overload encouraged
\end{lstlisting}
To allow chains of assignment, you should return a reference.
and operators return a reference to their left-hand argument.\\
Two reasons for this item:\\
\begin{enumerate}
  \item
  If you try return a reference to right-hand argument, compiler will disallow you to do this because the signiture has \emph{non-const} return value, which confilcts with right-hand argument. You if try to declare a signiture with \emph{non-const} argument. See reason 2
  \item
  \begin{lstlisting}
  String z = "Hello";
  \end{lstlisting}
  Compiler will translate to as follows. So that you have to declare a signiture with \emph{const} argument.
  \begin{lstlisting}
  const String temp("Hello");
  String z = temp;
  \end{lstlisting}
\end{enumerate}

\subsection{Check for assignment to self in operators}
The general assignment operator looks like this:\\
\begin{lstlisting}
C& C::operator= (const C& rhs) {
	if(this == &rhs) return *this;
	/* ... */ }
\end{lstlisting}
A more sophisticated mechanism for identity is implement the identity method yourself. See ~\ref{sec:inheritance_of_interface_and_implementation}

\section{Classes and Funcs: Design and Declare}



\section{Classes and Funcs: Implementation}

\section{Inheritance and OOD}

\section{Miscellany}

\rule{0.3\linewidth}{0.25pt}
\scriptsize

Copyright \copyright\ 2014 Jun Yu

\end{multicols}

\end{document}
