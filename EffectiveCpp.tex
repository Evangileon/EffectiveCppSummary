\documentclass[10pt,landscape]{article}
\usepackage{multicol}
\usepackage{calc}
\usepackage{ifthen}
\usepackage[letterpaper,landscape]{geometry}
\usepackage{hyperref}
\usepackage{titlesec}% http://ctan.org/pkg/titlesec

% To make this come out properly in landscape mode, do one of the following
% 1.
%  pdflatex latexsheet.tex
%
% 2.
%  latex latexsheet.tex
%  dvips -P pdf  -t landscape latexsheet.dvi
%  ps2pdf latexsheet.ps


\ifthenelse{\lengthtest { \paperwidth = 11in}}
	{ \geometry{top=.5in,left=.5in,right=.5in,bottom=.5in} }
	{\ifthenelse{ \lengthtest{ \paperwidth = 297mm}}
		{\geometry{top=1cm,left=1cm,right=1cm,bottom=1cm} }
		{\geometry{top=1cm,left=1cm,right=1cm,bottom=1cm} }
	}
	
%\geometry{top=1cm,left=1cm,right=1cm,bottom=1cm}

% Turn off header and footer
\pagestyle{empty}
 

% Redefine section commands to use less space
\makeatletter
\renewcommand{\section}{\@startsection{section}{1}{0mm}%
                                {-1ex plus -.5ex minus -.2ex}%
                                {0.5ex plus .2ex}%x
                                {\normalfont\large\bfseries}}
\renewcommand{\subsection}{\@startsection{subsection}{2}{0mm}%
                                {-1explus -.5ex minus -.2ex}%
                                {0.5ex plus .2ex}%
                                {\normalfont\normalsize\bfseries}}
\renewcommand{\subsubsection}{\@startsection{subsubsection}{3}{0mm}%
                                {-1ex plus -.5ex minus -.2ex}%
                                {1ex plus .2ex}%
                                {\normalfont\small\bfseries}}
\makeatother

% Numbering subsections continuously
\let\oldsection\section
\renewcommand{\section}{\oldsection*}
\renewcommand{\thesubsection}{\arabic{subsection}}

% Define BibTeX command
\def\BibTeX{{\rm B\kern-.05em{\sc i\kern-.025em b}\kern-.08em
    T\kern-.1667em\lower.7ex\hbox{E}\kern-.125emX}}

% Don't print section numbers
\setcounter{secnumdepth}{2}


\setlength{\parindent}{0pt}
\setlength{\parskip}{0pt plus 0.5ex}


% -----------------------------------------------------------------------

\begin{document}

\raggedright
\footnotesize

\begin{multicols}{3}

% multicol parameters
% These lengths are set only within the two main columns
%\setlength{\columnseprule}{0.25pt}
\setlength{\premulticols}{1pt}
\setlength{\postmulticols}{1pt}
\setlength{\multicolsep}{1pt}
\setlength{\columnsep}{2pt}

%\begin{center}
%     \Large{\textbf{Effective C++ Cheat Sheet}} \\
%\end{center}

\section{Shifting from C to C++}

\subsection{Prefer \texttt{const} and \texttt{inline} to \texttt{\#define}}

\subsection{Prefer \texttt{<iostream>} to \texttt{<stdio.h>}}

\subsection{Prefer \texttt{new} and \texttt{delete} to \texttt{malloc} and \texttt{free}}

\subsection{Prefer C++-style comment}

\section{Memory Management}

\subsection{Use same form of \texttt{new} and \texttt{delete}}
\begin{tabular}{@{}ll@{}}
\texttt{new class} & \texttt{delete *class} \\
\texttt{new class[]} & \texttt{delete [] class} \\
\end{tabular}

\subsection{Use delete on pointer members in dtors}
Avoid memory leak. \texttt{delete} a null pointer is always \emph{safe}

\subsection{Be prepared for \textsf{out-of-memory} conditions}
Always check whether the allocation succeed after \texttt{new}, using \texttt{try .. catch} block or \emph{assert NOT} null pointer.

\begin{verbatim}
#include <new>  // this handler invoked while fail to
typedef void (*new_handler)();  // allocate req memory
new_handler set_new_handler(new_handler p) throw();
\end{verbatim}

\begin{verbatim}
class X { // specify new_hanlder for a class
public:
    static new_handler set_new_handler(new_handler p);
    static void * operator new(size_t size);
private:
    static new_handler currentHandler;
} // This is also useful for Object Pool
\end{verbatim}

\subsection{Follow the convention of \texttt{new} and \texttt{delete}}
\begin{verbatim}
class Base { // user-define new and delete just for this class
    static void * operator new(size_t size);
    static void operator delete(void * rawMemory, size_t size);
} // compiler will fill size parameter automatically
\end{verbatim}

\subsection{You have to write normal form of \texttt{new}}
That means if you want to write a new handler, which has extra parameters other than "normal" form mention in last item, you have to write normal \texttt{new} first.

\subsection{Write \texttt{delete} if you write \texttt{delete}}



\section{Ctors, Dtors, and =}

\section{Classes and Funcs: Design and Declare}

\section{Classes and Funcs: Implementation}

\section{Inheritance and OOD}

\section{Miscellany}

\rule{0.3\linewidth}{0.25pt}
\scriptsize

Copyright \copyright\ 2014 Winston Chang

\end{multicols}

\end{document}
